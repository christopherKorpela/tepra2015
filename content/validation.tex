\section{System Validation and Results}\label{sec:validation}

To test and validate the valve turning framework, the system was implemented on a miniature gantry system. Dubbed Mini-SISTR, the gantry is modeled after the Systems Integrated Sensor Test Rig (SISTR) \cite{Korpela2014}. The experimental setup is shown in Fig. \ref{fig:gantry-quad}. The miniature gantry can traverse 0.35 m/s along each $x$, $y$, and $z$ axis. To provide yaw, pitch, and roll angles and velocities for the emulated aircraft, a 3-DOF gimbal is attached to the gantry z-axis. 

\begin{figure}
	\centering
	\includegraphics[width=0.45\textwidth]{./pictures/gantry.jpg}
	\caption{Mini-SISTR test and evaluation gantry rig.}
	\label{fig:gantry-quad}
\end{figure}

\subsection{Valve Turning Tests}

A series of valve turning experiments were performed for model verification. The valve is plastic and scaled to an actual industrial valve with an outer diameter of 15 $cm$. Fig. \ref{fig:valve-turn-test} shows a series of snapshots during a valve turning experiment. 

\begin{figure}[b]
	\centering
	{\subfloat[Detect valve using Hough transform. The aircraft-arm system aligns itself with valve.]{\includegraphics[width=0.35\textwidth]{pictures/shot1a.jpg}}
	\vfil
	\subfloat[Grab onto valve. The two arms articulate and grasp the valve until a desired joint torque is met. Note the orientation of the PVC pipe (left right).]{\includegraphics[width=0.35\textwidth]{pictures/shot2a.jpg}}
	\vfil
	\subfloat[Perform 1.57 radians turn. The yaw motion of the vehicle allows the valve to turn while the system is coupled. Note the new orientation of the PVC pipe (up down).]{\includegraphics[width=0.35\textwidth]{pictures/shot3a.jpg}}}
	\caption{Valve turning experiment while attached to the gantry test rig.}
	\label{fig:valve-turn-test}
\end{figure}

\subsection{Controller Performance}

Fig.~\ref{fig:plot1} shows the applied torques to the UAV yaw axis, right shoulder joint, right wrist joint, left shoulder joint, and left wrist joint of the two arms. When the arms make contact with the valve, the right and left shoulder joints experience the greatest applied torque throughout the turn. The wheel is rotated 1.57 radians in a clockwise direction with the quadrotor yaw torque shown in blue. The wheel is released at time stamp 25 sec. and then regrasped at time stamp 30 sec. The wheel is then rotated 1.57 radians in a counterclockwise direction. The applied torques by both shoulder and wrist joints provide a compliant and consist grasp during the valve rotation.

\begin{figure}
	\centering
      	{\subfloat[Quadrotor yaw torque and shoulder roll joint torques for both arms (Nm vs. time).]{\includegraphics[width=0.5\textwidth]{./pictures/shoulder.eps}}
	\vfil
      	\subfloat[Quadrotor yaw torque and wrist roll joint torques for both arms (Nm vs. time).]{\includegraphics[width=0.5\textwidth]{./pictures/wrist.eps}}}
	\caption{Applied torques during valve turning. The first grab and turn occurs between 10 and 25 seconds. The valve is released at time stamp 25 sec. and regrasped at time stamp 30 sec. for another rotation in the opposite direction.}
	\label{fig:plot1}
\end{figure}

%SISTR was developed as a hardware-in-the-loop test rig and designed to be used to evaluate obstacle detection sensors (LIDAR, computer vision, ultrasonic, ultra-wideband radar, millimeter wave radar, etc.), design sensor suites, and test collision avoidance algorithms.

%The miniature gantry can traverse 0.35 m/s along each $x$, $y$, and $z$ axis. To provide yaw, pitch, and roll angles and velocities for the emulated aircraft, a 3-DOF gimbal is attached to the gantry z-axis. Mini-SISTR can be tuned for almost any rotorcraft model in hover or near-hover modes. Models to account for ground-effect can also be incorporated to study perturbations and disturbances. For this particular analysis, the quadrotor model developed in Sec. \ref{sec:model} is implemented. The actual vehicle as described in Sec. \ref{sec:hardware} is 50 $cm$ in diameter and weighs 1.5 $kg$ where the gantry imitates the flight dynamics of this vehicle. 

%Using a recursive Newton-Euler algorithm and neglecting friction forces, one can derive generalized force/torque equations produced from each joint movement of the manipulators:

%\begin{equation}
%\sum_{j=0}^{n}[D_{ij}(\mathbf{q})\ddot{q}_j]+\sum_{k=0}^{n}\sum_{j=0}^{n}[C^i_{kj}(\mathbf{q})\dot{q}_k\dot{q}_j]+h_i(\mathbf{q})=\tau_i,0\leq i \leq n
%\label{eq:CompleteArmModel}
%\end{equation}
%with $D_{ij}$ as a generalized inertia tensor, $C^i_{kj}$ is the generalized Coriolis and Centrifugal force matrix and $h_i$ is a generalized gravity force. 

%Given that $\tau_0$ calculates forces produced on the aircraft body (i.e. $w=\tau_0$), Newton-Euler analysis provides the necessary tools to calculate static and dynamic disturbances acting on the rotorcraft. In a complete model, Newton-Euler equations for manipulator motion need to be provided with initial angular and linear speeds and accelerations. To simplify the overall problem, we make a reasonable assumption that the aircraft is in hover during manipulation. This assumption enables us to regard the initial linear and angular dynamics of the aircraft body as zero, thus effectively decoupling the dynamics.

%\subsection{Applied Torque Model}

%The 6-DOF miniature gantry utilizes a torque model to reproduce the reactions the aircraft undergoes when subjected to moments applied by the manipulator, interactions with the environment, or added load masses. The 3-axis gimbal provides yaw-pitch-roll position, velocity, torque, and impedance feedback and control. The applied torque is tuned according to the mass and inertial properties of the vehicle. A rotorcraft with a larger mass and inertia will be able to withstand larger pitch and roll moments compared to a vehicle with less mass and inertia. The basic assumption for the quadrotor in hover is that the thrust force and the force of gravity are equal:
%\begin{equation}
%\sum_{i=1}^{4}\vec{F}_i=m_Q\cdot \vec{g},
%\end{equation}
%where $\vec{q}$ represents the gravity acceleration vector. The quadrotor translates by tilting slightly in the desired direction. Therefore, for a small angle approximation, $x$ and $y$ coordinate dynamics can be derived. For clarity, we are showing only $x$ axis dynamics (i.e. force in the $x$ direction $F_x$ and respective acceleration $a_x$), produced from the quadrotor pitch angle $\Theta$. 
%\begin{equation}
%\begin{split}
%F_x =m\cdot g\,  \textup{sin}(\Theta) & \sim m_Q g \Theta\\
%a_x & \sim g\Theta
%\end{split}
%\end{equation}

%Because the aircraft interacts with the environment, the applied torques are fed into the simulated aircraft model. The attitude controller ultimately needs to compensate and correct these disturbances. Both static (constant torque applied by the manipulator mass) and dynamic (moving torques associated with articulation and environmental contact) are sensed by the attitude controller. This approach allows us to write a simple dynamic model for the emulation of quadrotor dynamics seen in Fig. \ref{fig:controller}.

%\begin{figure}
%	\centering
%	\includegraphics[width=0.5\textwidth]{./pictures/GantryAngleController.jpg}
%	\caption{Angle Attitude Controller (needs updating)}
%	\label{fig:controller}
%\end{figure}

%The disturbance force, $F_D$ is mapped from the applied torque, $\tau_D$, which contributes to the linear acceleration, $a_x$, of the vehicle base. $J$ is the inertial tensor for the 6-DOF aircraft. The linearized equation for $\tau_D \mapsto F_D$ mapping can be derived from the complete dynamic model of the arm in (\ref{eq:CompleteArmModel}). The acceleration resulting from the torque distance has the form:
%\begin{equation}
%	\centering
%	a_x = g\Theta + \frac{\tau_D \mapsto F_D }{J}
%\end{equation}
%Moments applied to the aircraft first introduce changes to angular positions and velocities which change the thrust vector. This altered thrust vector then causes a lateral translation in the vehicle. The torso joint positions adjust accordingly to emulate the pitching and rolling motions and the gantry provides the linear displacement in the $x$ and $y$ directions. Direct forces such as aerodynamic effects, propeller wash or ground effect, and wind gusts are not modeled. The forces and torques that do influence the vehicle pose are environmental contact through the end-effector, added load masses, and the mass and inertia of the manipulator itself.

%The second additive component to linear acceleration of the vehicle results from the projection of the gravity vector onto the $x$ or $y$ axis when the vehicle pitches or rolls. The angular position changes based on the linear velocity error and a proportional gain. As mentioned, the changing thrust vector due to a pitch or roll movement translates into a linear acceleration. 

%\subsection{Velocity Control Loop}

%The block diagram for the attitude controller is found in Fig. \ref{fig:controller}. 
%The reference velocity is set by the user through a joystick interface. This velocity is compared to the actual gantry velocity provided by a encoders on the gantry actuators. The error is fed into a PID controller which calculates the angular position required to translate the gantry and eliminate the error. The angle is projected onto the linear axis as acceleration through the gravity vector. The angular position of the aircraft changes in the same direction as the applied force. The measured torque, converted into a disturbance force and divided by the system mass, also contributes to the linear acceleration. An integrator converts acceleration into velocity which is sent to a gantry controller to actuate the gantry motors. Since the inertia properties are used to calculate the emulated UAV velocity, the controller can be tuned for almost any rotorcraft.
