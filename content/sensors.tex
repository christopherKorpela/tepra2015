\section{Sensors}\label{sec:sensors}

The power subsystem built into the package includes a switching regulator to step down 12V from the PayBreak interface chip into 5V to be used for the sensor package. The microcontroller is powered from the in-out side of the switching regulator because the microcontroller needs more voltage to ensure stable operation. The switching regulator can output over 2 Amps of current while maintaining over 4.75V. Theoretically, the power supply can deliver up to 10W of power.  The power supply can drive a large sensor load if required with little drop is supplied voltage. The prototype delivered 5.006V with no load and 4.986 with the full sensor load. This meets the requirements of the sensors. This meets the needs of the sensors. This power also meets the requirement of being powered by 2 BA-2590 batteries up to 2 hours because the batteries each have 14 Ah, with the full system operating, the current drain does not exceed 3A.  The power system used for the prototype can support much more powerful sensors in the future.

The Arduino microcontroller is used to process the sensor data and generate the data packets that are sent through the IP tables on the robot and wirelessly to the OCU to be stored in a database and displayed for the user. The code makes it possible for any user to modify, add, or subtract the current sensors used in order to fit the desired application. Furthermore, the paybreak housing pins are wired in order to allow access too many of the pins needed on the Arduino, which is shown Annex 5. The structure of the sensor packet is scalable for future years to add additional information. The received data was saved in a database to provide a history of the event's readings. We maintained the existing OCU to ensure the user does not have to learn a completely new control interface. A sensor toolbar was developed to mesh with the robot's current user interface to provide live data readings. When a sensor reading passes a dangerous threshold, the toolbar will change colors to alert the operator. Under each sensor, a button will display a graph of the sensor's history.

The sensor, power, and microcontroller subsystems all worked together to meet the interface, functional, and performance requirements of the completed system. Sensors were able to accurately detect the presence of a selection of harmful gasses, interface with the PackBot software to transmit and display data, and the system was able to store the data obtained on the mission. 