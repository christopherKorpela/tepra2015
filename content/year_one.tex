The specifications for the ARIBO-IH consist of three requirements: interface, functional, and performance requirements. The interface consisted of connecting a gas sensor package to a microcontroller. Our solution to this specification used an Ethernet connection, connecting the microcontroller to the Packbot via 40 pin ribbon cable provided with the PackBot, and connecting the PackBot to the OCU via 4.9 GHz radio. The functional requirements were to create a sensor package that could interface with the PackBot and to create a data capture system that could capture and record sensor data. Finally, the performance requirements for the ARIBO-IH were for the robot to detect several hazards to human health, to be able to operate in an indoor environment, to operate with sensors up to 2 hours without replacing batteries, and to be water-resistant for moist environments and decontamination. 

Since the gas sensors are inherently inaccurate due to their cheap construction and intended use for small projects, each sensor had to be calibrated by graphing the output of each sensor compared to the known gas in the testing environment. For the testing environments we used 1.0 liter sealed containers with the sensor inserted into one end and a septum for needles inserted in the other as illustrated by Figure 5. The 1.0 liter volume made calculating parts-per-million (ppm) simple and gas was introduced to the environment via a syringe and needle.