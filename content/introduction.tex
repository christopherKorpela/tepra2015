\section{Introduction}\label{sec:introduction}

Throughout the Army, Industrial Hygiene (IH) teams are responsible for inspection, environmental reconnaissance, and emergency response. IH is an integral part of installation force protection and is an important component of an installation’s toxic industrial chemical spill planned response. Current best practices for conducting the IH mission require direct human exposure to these hazardous environments. Robotic systems offer the potential to remove humans from these dangerous situations while maintaining the reliability and accuracy of the response team. Applying robotic solutions to this domain also contribute to the Department of Defense (DoD) unmanned systems goals outlined in the Unmanned Systems Integrated Roadmap FY2011-2036 \cite{roadmap}. Furthermore, robotic IH solutions are a force multiplier because these systems can be sent into a hazardous environment, parked, and allowed to collect data autonomously. Additionally, using robotic platforms for the IH environment is faster and safer than equipping and then decontaminating a human inspector. Finally, if successful, this project has potential DoD-wide and future civilian applications. 

\subsection{Trends}

Aging stocks of munitions and newly developed systems are creating larger quantities of dangerous materials that require constant monitoring and potentially, multiple emergency responses. In the current austere fiscal environment, enlarging a highly trained, professional IH team poses a significant challenge. Thus, it is more desirable to reuse/re-purpose existing inventory and to improve efficiency where possible.  One way to accomplish this goal is to automate repetitive tasks using currently available technologies such as ground robots and networked data acquisition systems. This automation can allow a smaller team of trained personnel to effectively as well as efficiently manage a larger amount of tasks that would traditionally require a large team.

\begin{figure}
	\centering
	\includegraphics[width=0.48\textwidth]{./pictures/concept}
	\caption{ARIBO-IH prototype vehicle inspecting a gas leak in a notional hazardous site. The inset in the lower left shows the camera view from the perspective of the robot.}
	\label{fig:concept}
\end{figure}

\subsection{Problems}

An Army IH Program mission is to reduce soldier and employee exposure to environmental factors and stresses including:  chemical (e.g., liquid, particulate dust, fumes, mist, vapor and gas), physical (e.g., electromagnetic radiation, temperature, ambient pressure, noise, vibration and ionizing radiation), and biological (e.g., agents of infectious diseases, insects, mites, molds, yeasts, fungi, bacteria and viruses) elements \cite{ArmymedPAM40503}. The majority of hazards come from industrial processes on Army installations. Army industrial hygiene personnel are at risk from exposure to these hazardous environments in the conduct of their duties. Additionally, rapidly equipping human teams for response to incidents and post-action decontamination pose difficult challenges.  

%\footnote{Tank Automotive Research, Development and Engineering Center}

\subsection{Benefits}

ARIBO (Applied Robotics for Installation and Base Operations) is a TARDEC-sponsored program to create ``living labs'' and leverage robotics technologies at military installations and civilian campuses. Through the use of robotic-enabled ground vehicles in a structured, controlled environment, the ARIBO-IH pilot will increase researchers\textquotesingle, manufacturers\textquotesingle, and users\textquotesingle \ understanding and familiarity of these systems in real-world operational scenarios. The ARIBO-IH pilot safely provides the service of IH inspections, and at the same time distancing the human IH professional from a potentially hazardous situation while reducing costs. Additionally, the project will facilitate the design, standardization, deployment, and supervision of the resulting ARIBO-IH inspection robots. Pilot locations include the Stanford Linear Accelerator Center (SLAC) National Accelerator Laboratory, Fort Bragg, North Carolina, and the the United States Military Academy (USMA), at West Point. By using cadets as researchers, they are exposed to Army technologies and systems at the beginning of their careers. The benefits of generating officers with technological backgrounds in robotics systems is paramount to achieving the DoD’s long term unmanned systems goals.

\subsection{Example Uses}
The robotic systems developed under this project could be employed in a number of situations to include:
\begin{itemize}
	\item Environmental reconnaissance in routine industrial hygiene tasks and emergency response
	%\item Weather station at the emission source
	\item Ventilation duct inspection
	\item Investigate suspected terrorist devices
	\item Site abatement or mitigation projects
\end{itemize}

This paper presents a solution to the industrial hygiene problem using a small ground vehicle equipped with a sensor package. The mobility platform for the system (Fig.~\ref{fig:concept}) is detailed in Sec.~\ref{sec:platform}. To allow for unattended operation in an unknown environment, Sec.~\ref{sec:ui} describes the user interface to remotely monitor sensor readings and toggle between teleoperation and autonomous modes. Sec.~\ref{sec:sensors} describes the sensor package used to perform the IH mission with sensor testing and results found in Sec.~\ref{sec:results}. Navigation and path planning are described in Sec.~\ref{sec:navigation}.
