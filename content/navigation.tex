\section{Navigation}\label{sec:navigation}

The internal computer of the GVR-Bot passes sensor data to the on-board, external embedded computer for processing, map generation, and development of a navigation scheme. The external embedded computer is directly connected and receives updates from the LiDar in real-time. The tilting capability of the LiDar allows 3-D point cloud data to be collected for use in developing a 3-D navigation solution. For the purposes of this proposal, accurate inertial measurements were not a requirement because navigation occurs in highly structured indoor environments, and odometry data helps mitigate errors in position. The computer then uses the laser data to generate a 2-D planar occupancy grid map. The technique of using occupancy grid maps based on LiDar scans is being implemented because it can be done at a low cost computationally, and it is a proven approach for localizing a ground robot in a real-world setting\cite{Probablistic Robotics 2005 MIT Press}. The 2-D SLAM information is coupled with 3-D pose estimate that is generated using EKF. This coupling allows the 2-D planar map to be overlayed with robot pose data estimated via EKF. 

This information is locally stored in memory for purposes of future unplanned events that may occur during the mission, such as connection loss. The highly detailed 2-D planar map is downsampled and a coarse version is created for the purposes of wireless transmission to the control unit to be displayed on the GUI for the user. The high computation capability of the computer allows for both the SLAM and 3D EKF components to be run on soft real-time with a low enough latency in calculation that it is negligible compared to actual or hard real-time 3-D navigation scheme; for the purposes of this proposal.